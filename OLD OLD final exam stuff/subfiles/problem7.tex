\documentclass[../CryptoFinal.tex]{subfiles}

\begin{document}
\begin{flushleft}



\numbpr{7}
\prob{7}  We start with values for $N$ and $\phi \left( N \right)$.  For clarity, we will denote the numerical value for $\phi \left( N \right)$ by the symbol $\Phi_N$.  Further, we know both that $N = p q$ and 

\begin{align}
  \phi \left( N \right) &= \phi \left( p \right) \, \phi \left( q \right) \notag \\
  &= \left( p - 1 \right) \left( q - 1 \right) \notag \\
  &= pq - p - q + 1 = \Phi_N \label{eq7a}
\end{align}

Additionally, note that the result in \ref{eq7a} was obtained using the relations $\phi \left( p \right) = p - 1$ and $\phi \left( q \right) = q - 1$.  The result in \ref{eq7a}, along with $N = p q$, means that we have the system of equations

\begin{align}
  \Phi_N = pq - p - q + 1 \tag{\ref{eq7a}}
\end{align}

and

\begin{align}
  N = p \, q \label{eq7b}
\end{align}

Rewriting the expression in \ref{eq7b} as $N = p \, q \; \Rightarrow \; q = N / p$ and applying the result, along with $N = p \, q$ to the expression in \ref{eq7a}, we have

\begin{align}
  \Phi_N &= N - p - \frac{N}{p} + 1 \notag \\
  p \, \Phi_N &= p \, N - p^2 - N + p \notag \\
  0 &= p^2 + \left( \Phi_N - N - 1 \right) p + N \label{eq7c}
\end{align}

which is solvable for $p$ in polynomial time (using the quadratic formula).  Applying the result from solving \ref{eq7c} for $p$ to the expression in \ref{eq7b} yields a value for $q$ in polynomial time as well.
























\end{flushleft}
\end{document}
