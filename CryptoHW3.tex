\documentclass{article}[12]
%\setlength{\textheight}{8.4in}
%\setlength{\topmargin}{-0.15in}
%\setlength{\oddsidemargin}{0in}
%\setlength{\evensidemargin}{0in}
%\setlength{\textwidth}{6.5in}

\setlength{\textheight}{8.75in}
\setlength{\topmargin}{-.5in}
\setlength{\oddsidemargin}{-0.25in}
\setlength{\evensidemargin}{0in}
\setlength{\textwidth}{7in}
\usepackage{amsfonts, amsmath, amsthm, amssymb,mathrsfs}
\usepackage{graphicx}
\usepackage{fancyhdr}
\usepackage{setspace}
\usepackage{xcolor}
\usepackage{mathtools}
\usepackage[]{algorithm2e}
\usepackage{algorithmicx}
\usepackage{multicol}
\usepackage{subfiles}
\usepackage{tgbonum}
\pagestyle{fancy}
\lhead{TCSS 581 - Autumn 2016}
\chead{Homework 2}
\rhead{Kamatchi S., Samir S., Jonathan M.}
\headsep = 22pt 
\headheight = 15pt

\doublespacing

% BEGIN PRE-AMBLE


% Setup equation numbering 
\numberwithin{equation}{section} 

%Equation Numbering Shortcut Commands
\newcommand{\numbch}[1]{\setcounter{section}{#1} \setcounter{equation}{0}}
%\newcommand{\numbpr}[1]{\setcounter{subsection}{#1} \setcounter{equation}{0}}
\newcommand{\numbpr}[1]{\setcounter{section}{#1} \setcounter{equation}{0}}
\newcommand{\note}{\textbf{NOTE:  }}

%Formatting shortcut commands
\newcommand{\chap}[1]{\begin{center}\begin{Large}\textbf{\underline{#1}}\end{Large}\end{center}}
\newcommand{\prob}[1]{\textbf{\underline{Problem #1):}}}
\newcommand{\sol}[1]{\textbf{\underline{Solution #1):}}}
\newcommand{\MMA}{\emph{Mathematica }}

%Text Shortcut Command
\newcommand{\s}[1]{\emph{Side #1}}

% Math shortcut commands
\newcommand{\dd}[2]{\frac{d #1}{d #2}}
\newcommand{\ddn}[3]{\frac{d^{#1} #2}{d #3^{#1}}}
%\newcommand{\dd}[2]{\frac{\textrm{d} #1}{\textrm{d} #2}}
%\newcommand{\ddn}[3]{\frac{\textrm{d}^{#1} #2}{\textrm{d} #3^{#1}}}
\newcommand{\pd}[2]{\frac{\partial #1}{\partial #2}}
\newcommand{\pdn}[3]{\frac{\partial^{#1} #2}{\partial #3^{#1}}}
\newcommand{\infint}{\int_{-\infty}^\infty}
\newcommand{\infiint}{\iint_{-\infty}^\infty}
\newcommand{\infiiint}{\iiint_{-\infty}^\infty}
\newcommand{\dint}[2]{\int_{#1}^{#2}}
\newcommand{\intdd}[1]{\textrm{d}#1}
\newcommand{\intddd}[1]{\textrm{d}#1}
\newcommand{\R}{\mathbb{R}}
\newcommand{\N}{\mathbb{N}}
\newcommand{\Z}{\mathbb{Z}}
%\newcommand{\mat}[1]{\overleftrightarrow{\mathbf{#1}}}
%\newcommand{\mat}[1]{\bar{\bar{\mathbf{#1}}}}
\newcommand{\mat}[1]{\overline{\overline{\mathbf{#1}}}}

%Math Text
\newcommand{\rect}{\text{ rect}}
\newcommand{\csch}{\text{ csch}}

%Physics Shortcut Commands
\newcommand{\h}{\mathcal{H}}


%MRI Stuff Shortcut Commands
\newcommand{\tno}{t_{n}}
\newcommand{\tn}[1]{t_{n#1}}
\newcommand{\Mno}{\vec{M}^{\left( n \right)}}
\newcommand{\Mn}[1]{\vec{M}^{\left( n #1 \right)}}
\newcommand{\Mnto}[1]{\vec{M}^{(n)} \left( t_{n} #1 \right)}
\newcommand{\Mnt}[2]{\vec{M}^{(n #1)} \left( t_{n #1} #2 \right)}
\newcommand{\rot}[2]{\mat{R}_{#1} \left( #2 \right)}
\newcommand{\DnMat}[2]{\mat{D} \left( t_{n #1} #2 \right)}
\newcommand{\rotINV}[2]{\mat{R}^{-1}_{#1} \left( #2 \right)}
\newcommand{\DnMatINV}[2]{\mat{D}^{-1} \left( t_{n #1} #2 \right)}
\newcommand{\betaNt}[2]{\beta \left( t_{n #1} #2 \right)}
\newcommand{\TR}{\textrm{TR}}


% Math formatting commands
\newcommand{\stack}[2]{\stackrel{\mathclap{\normalfont\mbox{#1}}}{#2}}


% 


% END PRE-AMBLE



\begin{document}
\begin{flushleft}



\numbpr{1}
\prob{1} Let $x, y, e, x^{-1} \in \mathcal{G}$ where $e \in \mathcal{G}$ is the identity element of $\mathcal{G}$ and $x^{-1}$ is such that both $x \, x^{-1} = e = x^{-1} \, x$ and $y \, x^{-1} = e = x^{-1} \, y$ hold.  Therefore we have 

\begin{align}
  x \, x^{-1} &= y \, x^{-1} \label{eq1a} \\
  x \, x^{-1} &= x^{-1} \, y \label{eq1b} \\
  x^{-1} \, x &= y \, x^{-1} \label{eq1c} \\
  x^{-1} \, x &= x^{-1} \, y \label{eq1d} 
\end{align}

By applying the cancelation rule ($a b = a c \; \Rightarrow \; b = c$ for $a, b, c \in \mathbb{G}$ for any group $\mathbb{G}$) to the expression in \ref{eq1a} and \ref{eq1d}, it is clear that we have

\begin{align}
  x = y \label{eq1e}
\end{align}

Since $\mathcal{G}$ is abelian, we may rewrite the expression in \ref{eq1b} as

\begin{align*}
  x \, x^{-1} = x^{-1} \, x = x^{-1} \, y
\end{align*}

or

\begin{align*}
  x \, x^{-1} = y \, x^{-1} = x^{-1} \, y
\end{align*}

From either expression, the application of the cancelation rule yields the same result as in expression \ref{eq1e}.  Similarly, we use the abelian property of $\mathcal{G}$ to rewrite the expression in \ref{eq1c} as

\begin{align*}
  x^{-1} \, x = x \, x^{-1} = y \, x^{-1}
\end{align*}

or

\begin{align*}
  x^{-1} \, x = x^{-1} \, y = y \, x^{-1}
\end{align*}

Again, applying the cancelation rule to either expression yields the same result as in \ref{eq1e}.  Therefore, every element in an abelian group must have a unique inverse. \\
$\square$



\vspace{0.33 in}



\numbpr{2}
\prob{2} Let $\mathcal{G}$ be a finite group and $g \in \mathcal{G}$.  Now define $\langle g \rangle \equiv g^0, g^1, g^2, \dots, g^k, \dots$, where $k \in \N$.  Beginning with the multiplicative case, let $m, n \in \N$ so that we have

\begin{align*}
  g^m \, g^n = g^{m + n}
\end{align*}

Since $m, n \in \N$ and $\N$ is closed under addition, $\left( m + n \right) \in \N$, it is clear that $g^{m + n} \in \langle g \rangle$.  Therefore, $\langle g \rangle$ is closed under its operation.  From our definition of $\langle g \rangle$, we know that $g^0 \in \langle g \rangle$.  Additionally, $g^0 \equiv e = 1$; therefore $\langle g \rangle$ contains the identity element.  Now, let $m \in \Z^+$ and write $g^{-m} \, g^m$.  Using $g^{-m} \equiv \left( g^{-1} \right)^m$, this yields

\begin{align*}
  g^{-m} \, g^m = \left( g^{-1} \right)^m \, g^m = \left( g^{-1} \, g \right)^m = \left( e \right)^m = e = 1
\end{align*}

which implies the existence of an inverse for each element in $\langle g \rangle$.  Finally, let $m, n, k \in \N$, then we have

\begin{align}
  g^m \, \left( g^n \, g^k \right) = g^m \, \left( g^{n + k} \right) = g^{m + \left( n + k \right)} \label{eq2a}
\end{align}

Since $\N$ is associative under addition, the expression in \ref{eq2a} may be rewritten as

\begin{align*}
  g^{m + \left( n + k \right)} = g^{\left( m + n \right) + k} = \left( g^{m + n} \right) \, g^k = \left( g^m \, g^n \right) \, g^k
\end{align*}

thereby demonstrating the associativity of operations in $\langle g \rangle$.  Since $\mathcal{G}$ is finite, it has order $m = \lvert \mathcal{G} \rvert$.  Therefore, the elements of $\langle g \rangle$ will be repeats of elements in $\mathcal{G}$ starting with $g^{m + 1}$.  Moreover, this means that $\langle g \rangle \subseteq \mathcal{G}$, thus satisfying the last condition for $\langle g \rangle$ to be a sub-group of $\mathcal{G}$. \newline

Continuing with the additive case, let $m, n \in \N$ so that we have

\begin{align*}
  m \times g \, n \times g = \left( m + n \right) \times g
\end{align*}

Since $m, n \in \N$ and $\N$ is closed under addition, $\left( m + n \right) \in \N$, it is clear that $\left(m + n\right) \times g \in \langle g \rangle$.  Therefore, $\langle g \rangle$ is closed under its operation.  From our definition of $\langle g \rangle$, we know that $0 \times g \in \langle g \rangle$.  Additionally, $0 \times g \equiv e = 0$; therefore $\langle g \rangle$ contains the identity element.  Now, let $m \in \Z^+$ and write $\left(-m\right) \times g \, m \times g$.  Using $\left(-m\right) \times g \equiv m \times \left( - g \right)^m$, this yields

\begin{align*}
  \left(-m\right) \times g \, m \times g = m \times \left( -g \right) \, m \times g = m \times \left( -g \, g \right) = m \times \left( e \right) = e = 0
\end{align*}

which implies the existence of an inverse for each element in $\langle g \rangle$.  Finally, let $m, n, k \in \N$, then we have

\begin{align}
  m \times g \, \left( n \times g \, k \times g \right) = m \times g \, \left( \left(n+k\right) \times g \right) = \left(m + \left( n + k \right)\right) \times g \label{eq2b}
\end{align}

Since $\N$ is associative under addition, the expression in \ref{eq2b} may be rewritten as

\begin{align*}
  \left(m + \left( n + k \right)\right) \times g = \left( \left(m+n\right) + k \right) \times g = \left(m+n\right) \times g \, k \times g = \left( m \times g \, n \times g \right) \, k \times g
\end{align*}

thereby demonstrating the associativity of operations in $\langle g \rangle$.  Since $\mathcal{G}$ is finite, it has order $m = \lvert \mathcal{G} \rvert$.  Therefore, the elements of $\langle g \rangle$ will be repeats of elements in $\mathcal{G}$ starting with $\left(m+1\right) \times g$.  Moreover, this means that $\langle g \rangle \subseteq \mathcal{G}$, thus satisfying the last condition for $\langle g \rangle$ to be a sub-group of $\mathcal{G}$.



\vspace{0.33 in}



\numbpr{3}
\prob{3} 

















































\end{flushleft}
\end{document}
