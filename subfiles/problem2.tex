\documentclass[../CryptoHW3.tex]{subfiles}

\begin{document}
\begin{flushleft}



\numbpr{2}
\prob{2} Let $\mathcal{G}$ be a finite group and $g \in \mathcal{G}$.  Now define $\langle g \rangle \equiv g^0, g^1, g^2, \dots, g^k, \dots$, where $k \in \N$.  Beginning with the multiplicative case, let $m, n \in \N$ so that we have

\begin{align*}
  g^m \, g^n = g^{m + n}
\end{align*}

Since $m, n \in \N$ and $\N$ is closed under addition, $\left( m + n \right) \in \N$, it is clear that $g^{m + n} \in \langle g \rangle$.  Therefore, $\langle g \rangle$ is closed under its operation.  From our definition of $\langle g \rangle$, we know that $g^0 \in \langle g \rangle$.  Additionally, $g^0 \equiv e = 1$; therefore $\langle g \rangle$ contains the identity element.  Now, let $m \in \Z^+$ and write $g^{-m} \, g^m$.  Using $g^{-m} \equiv \left( g^{-1} \right)^m$, this yields

\begin{align*}
  g^{-m} \, g^m = \left( g^{-1} \right)^m \, g^m = \left( g^{-1} \, g \right)^m = \left( e \right)^m = e = 1
\end{align*}

which implies the existence of an inverse for each element in $\langle g \rangle$.  Finally, let $m, n, k \in \N$, then we have

\begin{align}
  g^m \, \left( g^n \, g^k \right) = g^m \, \left( g^{n + k} \right) = g^{m + \left( n + k \right)} \label{eq2a}
\end{align}

Since $\N$ is associative under addition, the expression in \ref{eq2a} may be rewritten as

\begin{align*}
  g^{m + \left( n + k \right)} = g^{\left( m + n \right) + k} = \left( g^{m + n} \right) \, g^k = \left( g^m \, g^n \right) \, g^k
\end{align*}

thereby demonstrating the associativity of operations in $\langle g \rangle$.  Since $\mathcal{G}$ is finite, it has order $m = \lvert \mathcal{G} \rvert$.  Therefore, the elements of $\langle g \rangle$ will be repeats of elements in $\mathcal{G}$ starting with $g^{m + 1}$.  Moreover, this means that $\langle g \rangle \subseteq \mathcal{G}$, thus satisfying the last condition for $\langle g \rangle$ to be a sub-group of $\mathcal{G}$. \newline

Continuing with the additive case, let $m, n \in \N$ so that we have

\begin{align*}
  m \times g \, n \times g = \left( m + n \right) \times g
\end{align*}

Since $m, n \in \N$ and $\N$ is closed under addition, $\left( m + n \right) \in \N$, it is clear that $\left(m + n\right) \times g \in \langle g \rangle$.  Therefore, $\langle g \rangle$ is closed under its operation.  From our definition of $\langle g \rangle$, we know that $0 \times g \in \langle g \rangle$.  Additionally, $0 \times g \equiv e = 0$; therefore $\langle g \rangle$ contains the identity element.  Now, let $m \in \Z^+$ and write $\left(-m\right) \times g \, m \times g$.  Using $\left(-m\right) \times g \equiv m \times \left( - g \right)^m$, this yields

\begin{align*}
  \left(-m\right) \times g \, m \times g = m \times \left( -g \right) \, m \times g = m \times \left( -g \, g \right) = m \times \left( e \right) = e = 0
\end{align*}

which implies the existence of an inverse for each element in $\langle g \rangle$.  Finally, let $m, n, k \in \N$, then we have

\begin{align}
  m \times g \, \left( n \times g \, k \times g \right) = m \times g \, \left( \left(n+k\right) \times g \right) = \left(m + \left( n + k \right)\right) \times g \label{eq2b}
\end{align}

Since $\N$ is associative under addition, the expression in \ref{eq2b} may be rewritten as

\begin{align*}
  \left(m + \left( n + k \right)\right) \times g = \left( \left(m+n\right) + k \right) \times g = \left(m+n\right) \times g \, k \times g = \left( m \times g \, n \times g \right) \, k \times g
\end{align*}

thereby demonstrating the associativity of operations in $\langle g \rangle$.  Since $\mathcal{G}$ is finite, it has order $m = \lvert \mathcal{G} \rvert$.  Therefore, the elements of $\langle g \rangle$ will be repeats of elements in $\mathcal{G}$ starting with $\left(m+1\right) \times g$.  Moreover, this means that $\langle g \rangle \subseteq \mathcal{G}$, thus satisfying the last condition for $\langle g \rangle$ to be a sub-group of $\mathcal{G}$.



\end{flushleft}
\end{document}
