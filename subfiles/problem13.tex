\documentclass[../CryptoHW3.tex]{subfiles}

\begin{document}
\begin{flushleft}



\numbpr{13}
\prob{13} Storing the hash of the \emph{username} concatenated with the \emph{password} would not be considerably more secure than just storing the hashes of the passwords.  This is due to the fact that an attacker trying to crack passwords already has a list of usernames; therefore this would not significantly increase the amount of time required to check the stolen hashes against a precomputed table of hashes.  The only impediment is that the precomputed table would have to include hashes of longer strings, therefore possibly requiring more time to compute.






\end{flushleft}
\end{document}
